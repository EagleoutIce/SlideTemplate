\documentclass[
	aspectratio=169, % default is 43
	8pt, % font size, default is 11pt
	handout, % handout mode without animations, comment out to add animations
]{beamer}
\usetheme{uulm} % use the inofficial uulm beamer theme
%\usepackage[ngerman]{babel} % use this line for slides in German
%\recordingtrue % special recording mode for use with a greenscreen, gives you space to show yourself in a layer in front of the slides, has no effect in the handout mode

\title[Short Title]{This is the Long Title} % short title is used for the slide footer but optional
\subtitle[1.1 Short Subtitle]{1.1 This is the Long Subtitle} % subtitles are optional at all
\author[Short Author]{Long Author} % short author title is used for the slide footer but optional
\date{\today} % use a particular date here if needed

\begin{document}
    
\maketitle[titleimage][300] % optional parameters to change title picture and offset to move it upwards

\section{Lists and Boxes}

\subsection{Lists and Numbered Lists}
\begin{frame}{\insertsubsection}
	\leftandright{
	    Lists can be nested to a depth of three:
	    \begin{itemize}
	        \item Item on the first level
	    	\item Another item on the first level
	        \begin{itemize}
	            \item Item on the second level
	        	\item Another item on the second level
	            \begin{itemize}
	                \item Item on the third level
	                \item Another item on the third level
	            \end{itemize}
	        \end{itemize}
	    \end{itemize}
    }{
		Numbered lists can be nested to a depth of three:
		\begin{enumerate}
			\item Item on the first level
			\item Another item on the first level
			\begin{enumerate}
				\item Item on the second level
				\item Another item on the second level
				\begin{enumerate}
					\item Item on the third level
					\item Another item on the third level
				\end{enumerate}
			\end{enumerate}
		\end{enumerate}
	}
\end{frame}

\subsection{Colored Boxes}
\begin{frame}{\insertsubsection}
	\leftmiddleandright{
		\mydefinition{A Definition}{This is a definition.}
	}{
		\myexample{An Example}{This is an example.}
	}{
		\mynote{A Note}{This is a note.}
	}
\end{frame}

\section{Slide Layouts}

\subsection{Left and Right}
\begin{frame}{\insertsubsection}
    \leftandright{
        This is an example text that is shown in the \textbf{left column}.
    }{
        This is an example text that is shown in the \textbf{right column}.
    }
	\vfill
	\mynote{Explanation}{
		Both columns are visible in \textbf{handout}, \textbf{slide}, and \textbf{recording} mode (i.e., there are no animations).
	}
\end{frame}

\subsection{Left, Middle and Right}
\begin{frame}{\insertsubsection}
    \leftmiddleandright{
        This is an example text that is shown in the \textbf{left column}. 
    }{
        This is an example text that is shown in the \textbf{middle column}.
    }{
        This is an example text that is shown in the \textbf{right column}.
    }
	\vfill
	\mynote{Explanation}{
		All columns are visible in \textbf{handout}, \textbf{slide}, and \textbf{recording} mode (i.e., there are no animations).
	}
\end{frame}

\subsection{Left then Right}
\begin{frame}{\insertsubsection}
    \leftthenright{
        This is an example text that is shown in the \textbf{left column}.
    }{
        This is an example text that is shown in the \textbf{right column}.
    }
	\vfill
	\mynote{Explanation}{
		In \textbf{handout} mode, both columns are visible.

		In \textbf{slide} and \textbf{recording} mode, only the left column is shown at the beginning, then additionally the middle column, and finally all columns.
	}
\end{frame}

\subsection{Left, Middle then Right}
\begin{frame}{\insertsubsection}
    \leftmiddlethenright{
        This is an example text that is shown in the \textbf{left column}.
    }{
        This is an example text that is shown in the \textbf{middle column}.
    }{
        This is an example text that is shown in the \textbf{right column}.
    }
	\vfill
	\mynote{Explanation}{
		In \textbf{handout} mode, all columns are visible.

		In \textbf{slide} and \textbf{recording} mode, only the left column is shown at the beginning, then additionally the middle column, and finally all columns.
	}
	
\end{frame}

\subsection{Left or Right}
\begin{frame}{\insertsubsection}
    \leftorright{
        This is an example text that is shown in the \textbf{left column}.
    }{
        This is an example text that is shown in the \textbf{right column}.
    }
	\vfill
	\mynote{Explanation}{
		In \textbf{handout mode}, both columns are visible.
	
		In \textbf{slide mode}, only the left column is shown at the beginning and then both columns (cf. \textbf{Left then Right}).
		
		In \textbf{recording mode}, only the left column is shown at the beginning, then an empty slide (to walk to the other side), and finally only the right column.
	}
	
\end{frame}

\subsection{Left, Middle or Right}
\begin{frame}{\insertsubsection}
    \leftmiddleorright{
        This is an example text that is shown in the \textbf{left column}.
    }{
        This is an example text that is shown in the \textbf{middle column}.
    }{
        This is an example text that is shown in the \textbf{right column}.
    }
	\vfill
	\mynote{Explanation}{
		In \textbf{handout mode}, all columns are visible.
	
		In \textbf{slide mode}, only the left column is shown at the beginning, then additionally the middle column, and finally all columns (cf. \textbf{Left, Middle then Right}).
		
		In \textbf{recording mode}, only the left column is shown at the beginning, then only the middle column, and finally only the right column (again interleaved with empty slides).
	}
\end{frame}

\section{Other Features}

\subsection{Auto-Scaling for Long Titles}
\begin{frame}{\insertsubsection\ -- Long Titles are Scaled Down to the Available Space}
	\mynote{Explanation}{
		Very long frame titles are scaled down automatically.
		
		This can avoid annoying linebreaks for a single character or word.
		
		Use with care.
	}
\end{frame}

\subsection{Easy Navigation in Slides}
\begin{frame}{\insertsubsection}
	\vfill
	Click on the title (here \textbf{\insertshorttitle}) or subtitle (here \textbf{\insertshortsubtitle}) in the slide footer to jump to the title slide.
	\vfill
	Click on the section title (here \textbf{\insertsection}) in the slide footer to jump to the section overview.
	\vfill
	In the section overview, click on sections or subsections to jump to another part.
	\vfill
\end{frame}

\begin{frame}{\inserttitle}
	\vfill
	\tableofcontents
\end{frame}

\begin{frame}{List Spacing Test}
	Hello World!
	\begin{itemize}
		\item Item 1
		\item Item 2
		\item Item 3
	\end{itemize}
	Hello World!

	Hello World!
\end{frame}

\begin{frame}{Spacings inside leftandright}
	\leftandright{
		Hello World!
		\begin{itemize}
			\item Item 1
			\item Item 2
			\item Item 3
		\end{itemize}
		Hello World!

		Hello World!
	}{
		Hello World!
		\begin{enumerate}
			\item Item 1
			\item Item 2
			\item Item 3
		\end{enumerate}
		Hello World!

		Hello World!
	}
\end{frame}

\begin{frame}{Spacings inside tcolorbox}
	\mynote{Test}{
		Hello World!
		\begin{enumerate}
			\item Item 1
			\item Item 2
			\item Item 3
		\end{enumerate}
		Hello World!

		Hello World!
	}
\end{frame}

\begin{frame}{Paragraph Spacing Test}
	Lorem ipsum dolor sit amet, consetetur sadipscing elitr, sed diam nonumy eirmod tempor invidunt ut labore et dolore magna aliquyam erat, sed diam voluptua. At vero eos et accusam et justo duo dolores et ea rebum. Stet clita kasd gubergren, no sea takimata sanctus est Lorem ipsum dolor sit amet. Lorem ipsum dolor sit amet, consetetur sadipscing elitr, sed diam nonumy eirmod tempor invidunt ut labore et dolore magna aliquyam erat, sed diam voluptua. At vero eos et accusam et justo duo dolores et ea rebum. Stet clita kasd gubergren, no sea takimata sanctus est Lorem ipsum dolor sit amet.

	Lorem ipsum dolor sit amet, consetetur sadipscing elitr, sed diam nonumy eirmod tempor invidunt ut labore et dolore magna aliquyam erat, sed diam voluptua. At vero eos et accusam et justo duo dolores et ea rebum. Stet clita kasd gubergren, no sea takimata sanctus est Lorem ipsum dolor sit amet. Lorem ipsum dolor sit amet, consetetur sadipscing elitr, sed diam nonumy eirmod tempor invidunt ut labore et dolore magna aliquyam erat, sed diam voluptua. At vero eos et accusam et justo duo dolores et ea rebum. Stet clita kasd gubergren, no sea takimata sanctus est Lorem ipsum dolor sit amet.
\end{frame}
\lectureoverview

\end{document}