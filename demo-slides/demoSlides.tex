\documentclass[
	aspectratio=169, % default is 43
	8pt, % font size, default is 11pt
	%nosectionframes, % disable automatic frames at the begin of each section (default: sectiontitleslides, sectionoverviews in handout mode)
	%sectiontitleslides, % enable an automatic section title slide at the begin of each section
	%sectionoverviews, % enable an automatic section overview at the begin of each section
	%handout, % handout mode without animations, comment out to add animations
]{beamer}

\usepackage{../beamerthemeuulm} % use the inofficial uulm beamer theme
\setfaculty{infIngPsy} % set the color scheme for your faculty here [med/infIngPsy/math/nat]

%\institutelogo{sp} % set the institute logo
%\universitylogo{uulm} % set a new university logo
%\clearuniversitylogo % clear existing university logo
%\clearinstitutelogo % clear existing institute logo
%\uulmlogos{sp,uulm} % freely configure multiple logos (overwrites any other logo setting)

%\usepackage[ngerman]{babel} % use this line for slides in German
%\recordingtrue % special recording mode for use with a greenscreen, gives you space to show yourself in a layer in front of the slides, has no effect in the handout mode

%\setmycolumnsdefault{margin=7mm,t} % change the default for 'mycolumns' environment (default: margins 7mm, top-aligned)

%\includeonlyframes{current} % default mechanism of beamer to include only labeled frames, can be used for debugging or drafting

\title[Short Title]{This is the Long Title} % short title is used for the slide footer but optional
\subtitle[Short Subtitle]{This is the Long Subtitle} % subtitles are optional at all
\author[Short Author]{Long Author} % short author title is used for the slide footer but optional
\date{\today} % use a particular date here if needed

\begin{document}

\maketitle % title page with default picture
\maketitle[] % title page without a picture
% examples for alternative title pictures:
\maketitle[mar22-bee1][0] % optional parameters to change title picture and offset to move it upwards
\maketitle[may21-north1][50] % optional parameters to change title picture and offset to move it upwards
\maketitle[oct20-o25b][100] % optional parameters to change title picture and offset to move it upwards
\maketitle[may21-west1][150] % optional parameters to change title picture and offset to move it upwards
\maketitle[may21-south][175] % optional parameters to change title picture and offset to move it upwards
\maketitle[may21-q47][200] % optional parameters to change title picture and offset to move it upwards
\maketitle[oct20-o27c][250] % optional parameters to change title picture and offset to move it upwards
\maketitle[may21-ulm][300] % optional parameters to change title picture and offset to move it upwards
\maketitle[oct20-q37][350] % optional parameters to change title picture and offset to move it upwards

\section{Lists and Boxes}

\subsection{Lists and Numbered Lists}
\begin{frame}[label=current]{\insertsubsection}
	\leftandright{
	    Lists can be nested to a depth of three:
	    \begin{itemize}
	        \item Item on the first level
	    	\item Another item on the first level
	        \begin{itemize}
	            \item Item on the second level
	        	\item Another item on the second level
	            \begin{itemize}
	                \item Item on the third level
	                \item Another item on the third level
	            \end{itemize}
	        \end{itemize}
	    \end{itemize}
    }{
		Numbered lists can be nested to a depth of three:
		\begin{enumerate}
			\item Item on the first level
			\item Another item on the first level
			\begin{enumerate}
				\item Item on the second level
				\item Another item on the second level
				\begin{enumerate}
					\item Item on the third level
					\item Another item on the third level
				\end{enumerate}
			\end{enumerate}
		\end{enumerate}
	}
\end{frame}

\subsection{Colored Boxes}
\begin{frame}{\insertsubsection}
	Normal versions:
	\begin{mycolumns}[columns=3]
		\begin{definition}{A Definition}
			This is a definition.
		\end{definition}
	\mynextcolumn
		\begin{example}{An Example}
			This is an example.
		\end{example}
	\mynextcolumn
		\begin{note}{A Note}
			This is a note.
		\end{note}
	\end{mycolumns}
	\vfill
	Tight versions (e.g., for use with pictures):
	\begin{mycolumns}[columns=3]
		\begin{definitiontight}{A Definition}
			This is a definition.
		\end{definitiontight}
	\mynextcolumn
		\begin{exampletight}{An Example}
			This is an example.
		\end{exampletight}
	\mynextcolumn
		\begin{notetight}{A Note}
			This is a note.
		\end{notetight}
	\end{mycolumns}
\end{frame}

\section{Slide Layouts}

\subsection{Multiple Columns}
\begin{frame}{\insertsubsection}
	\begin{mycolumns}[columns=3] % default: 2 columns
		This is an example text that is shown in the \textbf{first column}.
	\mynextcolumn
		This is an example text that is shown in the \textbf{second column}.
	\mynextcolumn
		This is an example text that is shown in the \textbf{third column}.
	\end{mycolumns}
	\vfill
	\begin{note}{Explanation}
		Columns are separated by the command \texttt{\textbackslash mynextcolumn}.

		The option \texttt{columns} specifies the number of columns that should be generated. The default number of columns is two.
	\end{note}
\end{frame}

\subsection{Animations (keep)}
\begin{frame}{\insertsubsection}
	\begin{mycolumns}[keep] % short for animation=keep
		This is an example text that is shown in the \textbf{first column}.
	\mynextcolumn
		This is an example text that is shown in the \textbf{second column}.
	\end{mycolumns}
	\vfill
	\begin{note}{Explanation}
		The option \texttt{keep} (short for \texttt{animation=keep}) introduces an animation which lets the columns appear one after another.
		The previous columns are kept on the slide when a new column is displayed.
	\end{note}
\end{frame}

\subsection{Animations (forget)}
\begin{frame}{\insertsubsection}
	\begin{mycolumns}[forget] % short for animation=forget
		This is an example text that is shown in the \textbf{first column}.
	\mynextcolumn
		This is an example text that is shown in the \textbf{second column}.
	\end{mycolumns}
	\vfill
	\begin{note}{Explanation}
		The option \texttt{forget} (short for \texttt{animation=forget}) introduces an animation which lets the columns appear one after another.
		The previous columns are removed when a new column is displayed.

		This option only works in recording mode, otherwise it will behave the same as \texttt{keep}.
	\end{note}
\end{frame}

\subsection{Direction of Animations}
\begin{frame}{\insertsubsection}
	\begin{mycolumns}[keep,reverse]
		This is an example text that is shown in the \textbf{first column}.
	\mynextcolumn
		This is an example text that is shown in the \textbf{second column}.
	\end{mycolumns}
	\vfill
	\begin{note}{Explanation}
		The option \texttt{reverse} inverts the order of any animation that is specified (\texttt{keep} or \texttt{forget}).
	\end{note}
\end{frame}

\subsection{Columns with Customized Width}
\begin{frame}{\insertsubsection}
	\begin{mycolumns}[columns=3,widths={20,30,50}]
		\begin{example}{}
			The \textbf{first column} takes 20 \% of the width of the slide.
		\end{example}
	\mynextcolumn
		\begin{example}{}
			The \textbf{second column} takes 30 \% of the width of the slide.
		\end{example}
	\mynextcolumn
		\begin{example}{}
			The \textbf{third column} takes 50 \% of the width of the slide.
		\end{example}
	\end{mycolumns}
	\vfill
	\begin{note}{Explanation}
		The widths of the single columns can also be manually specified by a list of percentages given to the option \texttt{\textbackslash widths}.
	\end{note}
\end{frame}

\section{Old Slide Layouts (Deprecated)}

\begin{frame}{\insertsection}
	\begin{note}{Note}
		The following slide layouts are replaced by the \texttt{mycolumns}-envrionment and therefore deprecated.

		Please do not use them anymore, as they are going to be removed from the template in the future.
	\end{note}
\end{frame}

\subsection{Left and Right}
\begin{frame}{\insertsubsection}
    \leftandright{
        This is an example text that is shown in the \textbf{left column}.
    }{
        This is an example text that is shown in the \textbf{right column}.
    }
	\vfill
	\begin{note}{Explanation}
		Both columns are visible in \textbf{handout}, \textbf{slide}, and \textbf{recording} mode (i.e., there are no animations).
	\end{note}
\end{frame}

\subsection{Left, Middle, and Right}
\begin{frame}{\insertsubsection}
	\leftmiddleandright{
		This is an example text that is shown in the \textbf{left column}.
	}{
		This is an example text that is shown in the \textbf{middle column}.
	}{
		This is an example text that is shown in the \textbf{right column}.
	}
	\vfill
	\begin{note}{Explanation}
		All columns are visible in \textbf{handout}, \textbf{slide}, and \textbf{recording} mode (i.e., there are no animations).
	\end{note}
\end{frame}

\subsection{Left then Right}
\begin{frame}{\insertsubsection}
    \leftthenright{
        This is an example text that is shown in the \textbf{left column}.
    }{
        This is an example text that is shown in the \textbf{right column}.
    }
	\vfill
	\begin{note}{Explanation}
		In \textbf{handout} mode, both columns are visible.

		In \textbf{slide} and \textbf{recording} mode, only the left column is shown at the beginning, then both columns.
	\end{note}
\end{frame}

\begin{frame}{Right then Left}
    \rightthenleft{
        This is an example text that is shown in the \textbf{left column}.
    }{
        This is an example text that is shown in the \textbf{right column}.
    }
	\vfill
	\begin{note}{Explanation}
		In \textbf{handout} mode, both columns are visible.

		In \textbf{slide} and \textbf{recording} mode, only the right column is shown at the beginning, then both columns.
	\end{note}
\end{frame}

\subsection{Left, Middle, then Right}
\begin{frame}{\insertsubsection}
    \leftmiddlethenright{
        This is an example text that is shown in the \textbf{left column}.
    }{
        This is an example text that is shown in the \textbf{middle column}.
    }{
        This is an example text that is shown in the \textbf{right column}.
    }
	\vfill
	\begin{note}{Explanation}
		In \textbf{handout} mode, all columns are visible.

		In \textbf{slide} and \textbf{recording} mode, only the left column is shown at the beginning, then additionally the middle column, and finally all columns.
	\end{note}
\end{frame}

\begin{frame}{Right, Middle, then Left}
    \rightmiddlethenleft{
        This is an example text that is shown in the \textbf{left column}.
    }{
        This is an example text that is shown in the \textbf{middle column}.
    }{
        This is an example text that is shown in the \textbf{right column}.
    }
	\vfill
	\begin{note}{Explanation}
		In \textbf{handout} mode, all columns are visible.

		In \textbf{slide} and \textbf{recording} mode, only the right column is shown at the beginning, then additionally the middle column, and finally all columns.
	\end{note}
\end{frame}

\subsection{Left or Right}
\begin{frame}{\insertsubsection}
    \leftorright{
        This is an example text that is shown in the \textbf{left column}.
    }{
        This is an example text that is shown in the \textbf{right column}.
    }
	\vfill
	\begin{note}{Explanation}
		In \textbf{handout mode}, both columns are visible.

		In \textbf{slide mode}, only the left column is shown at the beginning and then both columns (cf. \textbf{Left then Right}).

		In \textbf{recording mode}, only the left column is shown at the beginning, then an empty slide (to walk to the other side), and finally only the right column.
	\end{note}
\end{frame}

\begin{frame}{Right or Left}
    \rightorleft{
        This is an example text that is shown in the \textbf{left column}.
    }{
        This is an example text that is shown in the \textbf{right column}.
    }
	\vfill
	\begin{note}{Explanation}
		In \textbf{handout mode}, both columns are visible.

		In \textbf{slide mode}, only the right column is shown at the beginning and then both columns (cf. \textbf{Right then Left}).

		In \textbf{recording mode}, only the right column is shown at the beginning, then an empty slide (to walk to the other side), and finally only the left column.
	\end{note}
\end{frame}

\subsection{Left, Middle, or Right}
\begin{frame}{\insertsubsection}
    \leftmiddleorright{
        This is an example text that is shown in the \textbf{left column}.
    }{
        This is an example text that is shown in the \textbf{middle column}.
    }{
        This is an example text that is shown in the \textbf{right column}.
    }
	\vfill
	\begin{note}{Explanation}
		In \textbf{handout mode}, all columns are visible.

		In \textbf{slide mode}, only the left column is shown at the beginning, then additionally the middle column, and finally all columns (cf. \textbf{Left, Middle, then Right}).

		In \textbf{recording mode}, only the left column is shown at the beginning, then only the middle column, and finally only the right column (again interleaved with empty slides).
	\end{note}
\end{frame}

\begin{frame}{Right, Middle, or Left}
    \rightmiddleorleft{
        This is an example text that is shown in the \textbf{left column}.
    }{
        This is an example text that is shown in the \textbf{middle column}.
    }{
        This is an example text that is shown in the \textbf{right column}.
    }
	\vfill
	\begin{note}{Explanation}
		In \textbf{handout mode}, all columns are visible.

		In \textbf{slide mode}, only the right column is shown at the beginning, then additionally the middle column, and finally all columns (cf. \textbf{Right, Middle, then Left}).

		In \textbf{recording mode}, only the right column is shown at the beginning, then only the middle column, and finally only the left column (again interleaved with empty slides).
	\end{note}
\end{frame}

\section{Other Features}

\subsection{Auto-Scaling for Long Titles}
\begin{frame}{\insertsubsection\ -- Long Titles are Scaled Down to the Available Space}
	\begin{note}{Explanation}
		Very long frame titles are scaled down automatically.

		This can avoid annoying linebreaks for a single character or word.

		Use with care.
	\end{note}
\end{frame}

\subsection{Easy Navigation in Slides}
\begin{frame}{\insertsubsection}
	\vfill
	Click on the title (here \textbf{\insertshorttitle}) or subtitle (here \textbf{\insertshortsubtitle}) in the slide footer to jump to the title slide.
	\vfill
	Click on the section title (here \textbf{\insertsection}) in the slide footer to jump to the section overview.
	\vfill
	In the section overview, click on sections or subsections to jump to another part.
	\vfill
\end{frame}

\lectureoverview

\end{document}